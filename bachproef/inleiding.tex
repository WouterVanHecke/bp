%%=============================================================================
%% Inleiding
%%=============================================================================

\chapter{Inleiding}
\label{ch:inleiding}

Technologie blijft evolueren, dit geldt voor alle sectoren. In deze bachelorproef wordt er dieper ingegaan in de sector van supply chain, meer specifiek voor de agricultuur. Deze technologie begint te verouderen, maar is zeer moeilijk om zomaar te gaan vervangen. De huidige supply chain management systemen zoals ERP en SAP brengen een paar problemen met zich mee, problemen waar een blockchain technologie wel hulp kan bieden. Deze problemen worden onderzocht en er wordt op een praktische manier naar een oplossing gezocht.
\\*\\*
Als praktische oplossing wordt een een fictief netwerk opgesteld dat een supply chain representeerd. Op dit netwerk zijn er bedrijven zoals winkels en fabrieken te vinden, net zoals de producten die ze kopen/verkopen. Natuurlijk kan dit netwerk zo uitgebreid mogelijk gemaakt worden, maar voor dit prototype hebben we het gehouden rond het proces van tomaten.
\\*\\*
Voor een bedrijf dat dit netwerk zou gebruiken, is de snelheid ervan ook een belangrijk onderdeel. Daarom worden er testen gehouden op de snelheid van het netwerk. Deze testen gaan van het aanmaken van producten tot het ophalen van 1 enkel product.

\section{Probleemstelling}
\label{sec:probleemstelling}

De bedoeling van deze bachelorproef is om bedrijven dat gebruik maken van supply chain, een prototype te tonen van hoe een supply chain eruit kan zien met een blockchain technologie. Dit kan gaan van bedrijven zoals Colruyt, waar het heel belangrijk is, ook naar de eindgebruiker toe, van waar het product precies komt, onder welke omstandigheden dat het getransporteert werd en natuurlijk hoe lang het product onderweg was met als laatste een van de belangrijkste vragen: Hoe lang blijven deze producten nog goed?

Aan de andere kant is deze bachelorproef ook handig voor IT-developers dat actief zijn in supply chain technologie. Dankzij dit prototype hebben ze een duidelijk beginpunt, het is dus bedoeling dat er verder kan gewerkt worden bovenop dit protoype.

\section{Onderzoeksvraag}
\label{sec:onderzoeksvraag}

'Hoe kan de blockchain technologie de problemen oplossen in verband met supply chain in de agricultuur?' Dit is de hoofdonderzoeksvraag. Er wordt gekeken naar de huidige problemen van een supply chain en hoe blockchain een rol kan spelen in het oplossen van deze problemen of hoe blockchain een meerwaarde kan zijn.

Deze hoofdonderzoeksvraag wordt nog eens onderverdeeld in drie kleinere deelvragen:

\begin{enumerate}
	\item \textbf{Hoe kan er met zekerheid gezegd worden dat de producten authentiek zijn?}\\*
	Bij deze vraag wordt er dieper op ingegeaan hoe men met zekerheid kan zeggen dat er niet gebeurt is met de producten dat niet mag en vooral aantonen van fabrieken deze producten deze producten komen. Wat tegenwoordig een heel belangrijk details is voor mensen.
	\item \textbf{Heeft het aantal gegevens op het netwerk een invloed op het ophalen en wegsturen van informatie?}\\*
	Zoals al eerder aangehaald werd, is het voor een bedrijf zeer belangrijk dat het netwerk optimaal is. Daarom is het handig als er enkele testen worden uitgevoerd op het netwerk.
	\item \textbf{Welke problemen kunnen niet opgelost worden met blockchain?}\\*
	Blockchain is op zich ook nog in ontwikkeling en zeker nog ver van perfectie. Daarom is het ook belangrijk om aan te halen welke aspecten niet / nog niet mogelijk zijn om met blockchain te realiseren.
\end{enumerate}

\section{Onderzoeksdoelstelling}
\label{sec:onderzoeksdoelstelling}

Er worden een paar zaken verwacht van de resultaten:

Voor de bevolking is het tegengaan van vervalsing en zekerheid hebben over hun producten topprioriteit. Dit gebeurt de dag van vandaag zeker niet altijd. Maar blockchain kan hier zeker een verandering in brengen. Omdat eens aangebrachte veranderingen voor altijd op de blockchain staat, kan dit door de eindpartij ook zelf bekeken worden of het inderdaad klopt wat er precies op het label staat.\\*\\*Er wordt verwacht van de resultaten van de testen dat deze snel zal zijn en dat er relatief gezien niet veel verschil zal zitten tussen testen met een groter aantal gegevens.\\*\\*Omdat blockchain zelf nog in zijn beginfase zit, zullen er waarschijnlijk ook wel aspecten zijn dat het nog niet kan oplossen. Supply chain zit ook heel complex in elkaar. Er zal naar de toekomst toe zeker nog heel veel werk in kruipen naar gelang implementatie toe.

\section{Opzet van deze bachelorproef}
\label{sec:opzet-bachelorproef}

De rest van deze bachelorproef is als volgt opgebouwd:

In Hoofdstuk~\ref{ch:stand-van-zaken} wordt een overzicht gegeven van de stand van zaken binnen het onderzoeksdomein, op basis van een literatuurstudie.

In Hoofdstuk~\ref{ch:methodologie} wordt de methodologie toegelicht en worden de gebruikte onderzoekstechnieken besproken om een antwoord te kunnen formuleren op de onderzoeksvragen.

In Hoofdstuk 4 wordt een introductie gegeven over Bitcoin, dit is een belangrijke basis om verder te gaan met blockchain.

In Hoofdstuk 5 wordt een introductie gegeven over blockchain, hier worden belangrijke concepten aangehaald wat belangrijk zal zijn voor bepaalde vragen.

In Hoofdstuk 6 wordt er gekeken naar wat een supply chain precies inhoud en wat de belangrijke aspecten hier in zijn.

In Hoofdstuk 7 worden de problemen van een supply chain onderzocht en gekeken hoe de blockchain op die vlakken kunnen helpen.

In Hoofdstuk 8 wordt uitleg gegeven over het gemaakte prototype met hyperledger composer. Hierin wordt er vertelt hoe het netwerk is opgemaakt met een demonstratie van de gemaakte mobile applicatie.

In Hoofdstuk 9 worden de testen besproken dat los gelaten worden op het netwerk met de uiteindelijke conclusie van de resultaten.

In Hoofdstuk 10 wordt besproken wat er verder onderzocht kan worden in de richting van supply chain gecombineerd met blockchain / welke andere testen er eventueel nog kunnen uitgevoerd worden.

In Hoofdstuk~\ref{ch:conclusie}, tenslotte, wordt de conclusie gegeven en een antwoord geformuleerd op de onderzoeksvragen. Daarbij wordt ook een aanzet gegeven voor toekomstig onderzoek binnen dit domein.

