\chapter{Stand van zaken}
\label{ch:stand-van-zaken}

% Tip: Begin elk hoofdstuk met een paragraaf inleiding die beschrijft hoe
% dit hoofdstuk past binnen het geheel van de bachelorproef. Geef in het
% bijzonder aan wat de link is met het vorige en volgende hoofdstuk.

% Pas na deze inleidende paragraaf komt de eerste sectiehoofding.

\section{Blockchain + Supply chain}
In het eerste artikel ~\autocite{supchain1} wordt er informatie gegeven over wat de supply chain is en precies het doel is van supply chain management. Het geeft ook uitleg over wat de belemmeringen zijn van een supply chain en hoe de blockchain technologie in zijn tijd deze problemen kan helpen wegwerken. Een supply chain wordt omgeschreven als een sequentie van processen dat een product ondergaat. Een supply chain is alles behalve lineair, het is een netwerk van netwerken. De ene uitgang kan de ingang zijn van het volgende product. Maar het gaat niet alleen maar om de productstroom, ook over geldstroom, informatiestroom...\\*\\* Het belangrijkste van een supply chain is het beheren van de staat, dit kan gezien worden als de laatst geweten status van een proces of transactie. Deze dat wordt meestal ingegeven door meerdere bedrijven, zo kan het ontstaan het ene bedrijf informatie niet invult dat ergens anders wel ingevuld wordt. Zo onstaat er inconsistentie. Nog andere nadelen van de staat op dit moment is dat deze informatie niet real time zijn, er bestaat een vertraging tussen het uitwisselen van informatie tussen het digitale systeem en de fysieke supply chain. Deze informatie dat bijhouden wordt is heel bedrijfs centrisch gelegen. Er kan niet gekeken worden naar informatie buiten het eigen ERP systeem. Dit is namelijk het hoofdprobleem van ERP-systemen.\\*\\*Dit is alles behalve een netwerk van waarde, maar met de opkomst van de Blockchain technologie, samen met IoT en zelfs AI, kan dit allemaal veranderen in de toekomst. De huidige ERP systemen lijden aan een paar problemen dat kan worden opgelost met de blockchain technologie. Deze problemen zijn het manueel fouten maken bij het ingeven van gegevens en papierwerkfouten. De informatie ligt heel centrisch, maar het moet wel meerdere keren ingegeven worden door verschillende bedrijven. De aard van deze problemen ligt bij het feit dat bedrijven elkaar elkaar niet vertrouwen met hun informatie en daarom alles gecentreerd houden.\\*\\*Dankzij het opkomen van IoT, kan de informatie beheert en gecontroleerd worden in real time. Het automatiseren van processen kan vergemakkelijkt worden, terwijl dit wel voor meer transacties per seconden zorgt. De Blockchain technologie helpt daarnaast om één gedecentraliseerde database te verkrijgen, waar er transparantie ontstaat van de producten en processen. Dankzij de blockchain technologie kan er echt een netwerk van waarde worden opgesteld omdat de bedrijven elkaar niet hoeven te vertrouwen, het gedecentraliseerde netwerk zal het vertrouwen afdwingen.

\section{De agricultuur in China}
In het tweede artikel vertelt ~\textcite{agri1} over problemen in de agricultuur ter plaatse van China. Hij vertelt over de problemen van de boeren en de algemene problemen van de agricultuur. Het inkomen van de boeren is aan het zakken terwijl hun kosten alleen maar stijgen, terwijl de prijzen van het voedsel gedaald is en er minder vraag is van de middel klasse in China. De boeren hebben geen zekerheid tegen een grote ramp, zoals droogte en dat zo hun oogst niet zal slagen. Het land waarop ze werken is ook niet hun eigendom, ze huren dit van hun overheid. Hierdoor hebben ze weinig tot geen zeg over wat er moet gebeuren met het land in verband met verbeteringen. Ze hebben wel zelf de keuze wat ze planten, daar is er ook een moeilijkheid. Het is niet makkelijk om te voorspellen wat ze precies nodig hebben en hoeveel ze er van nodig hebben.\\*\\*Nu als het gaat over de gehele agricultuur in China, kan het volgende gezegd worden. Omdat de wegen en infrastructuur er zo slecht zijn, gebeurt het vaak dat de boer zijn producten niet op tijd bij de markten krijgt en ze daardoor rot aankomen. De kwaliteit van de producten zijn soms heel slecht of vervormd, door het gebruik van pesticiden en kunstmest. Vele landschappen zijn aan het uitdrogen of het landschap wordt onbruikbaar.

\section{Huidige problemen van een supply chain}
Het derde artikel ~\autocite{prob1} geeft een visie van welke huidige problemen er bestaan voor supply chain management. Maar eerst wordt er besproken wat een goede supply chain inhoud.\\*\\*Een supply chain moet de rauwe materialen kunnen bekijken, hierbij kan de supply chain manager het inplannen van nieuwe producten maken, effectieve inventaris opmaken om zoveel mogelijk kosten te verminderen en zich kunnen aanpassen aan de vraag van de consument. Alle informatie moet makkelijk verkrijgbaar zijn zodat er zich geen situaties voordoen als het moeten stopzetten van de productie. Een van de grootste problemen is dat de data real time moet zijn om een veel beter overzicht te krijgen van wat er allemaal gebeurt.\\*\\*Het eerste punt dat aangehaald wordt is de globalizering. Bedrijven verhuizen hun productiefabrieken meer naar het buitenland voor lagere kosten en minder belastingen, lagere transport kosten enzovoort. Maar dit brengt natuurlijk meer complexiteit binnen hun supply chain. De bedrijven moeten hun materialen meer organiseren en samen werken met andere bedrijven voor hun maken van producten, opslaan van het resultaat en het transport ervan.\\*\\*De markt is constant in beweging. Aankoop prijzen en verkoop prijzen veranderen constant, net zoals de vraag en het aanbod van de producten. De producten hebben een kortere levenscyclus, dit forceert de bedrijven om hun supply chain aan te passen. Het is moeilijk om te voorspellen hoeveel producten er precies moeten worden gemaakt, dat geldt dus ook voor hoeveel materiaal er moet worden aangekocht.\\*\\*Het laatste punt dat aangehaald wordt is de kwaliteit van de producten. Door de druk van sociale media moet er veel meer gekeken worden op consistente kwaliteit van de producten. Dit geldt voor het maken van het eindproduct, maar dit is ook belangrijk bij het aankopen van de rauwe materialen. De bedrijven moeten letten dat hun product tijdens de verwerking kwaliteitsvolle processen ondergaat, bv. tijdens het inpakken ervan, tijdens transportatie, zeker tijdens het kweken van eten zoals tomaten of granen.\\*\\*Tijdens deze processen moet er ook stevig wat papierwerk worden bijhouden. Dit is op zich al een heel makkelijk punt om fouten te maken. Een fout in het ingeven van gegevens kan al grote gevolgen hebben in het proces van het product. Er wordt ook rekening gehouden met documenten zoals vergunningen, licenties, certificaten.


%%Dit hoofdstuk bevat je literatuurstudie. De inhoud gaat verder op de inleiding, maar zal het onderwerp van de bachelorproef *diepgaand* uitspitten. De bedoeling is dat de lezer na lezing van dit hoofdstuk helemaal op de hoogte is van de huidige stand van zaken (state-of-the-art) in het onderzoeksdomein. Iemand die niet vertrouwd is met het onderwerp, weet er nu voldoende om de rest van het verhaal te kunnen volgen, zonder dat die er nog andere informatie moet over opzoeken \autocite{Pollefliet2011}.

%%Je verwijst bij elke bewering die je doet, vakterm die je introduceert, enz. naar je bronnen. In \LaTeX{} kan dat met het commando \texttt{$\backslash${textcite\{\}}} of \texttt{$\backslash${autocite\{\}}}. Als argument van het commando geef je de ``sleutel'' van een ``record'' in een bibliografische databank in het Bib\TeX{}-formaat (een tekstbestand). Als je expliciet naar de auteur verwijst in de zin, gebruik je \texttt{$\backslash${}textcite\{\}}.
%%Soms wil je de auteur niet expliciet vernoemen, dan gebruik je \texttt{$\backslash${}autocite\{\}}. In de volgende paragraaf een voorbeeld van elk.

%%\textcite{Knuth1998} schreef een van de standaardwerken over sorteer- en zoekalgoritmen. Experten zijn het erover eens dat cloud computing een interessante opportuniteit vormen, zowel voor gebruikers als voor dienstverleners op vlak van informatietechnologie~\autocite{Creeger2009}.
