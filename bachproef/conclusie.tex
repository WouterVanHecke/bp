%%=============================================================================
%% Conclusie
%%=============================================================================

\chapter{Conclusie}
\label{ch:conclusie}

De onderzoeksvragen worden snel nog eens hieronder geplaatst:
\begin{enumerate}
	\item \textbf{Hoe kan er met zekerheid gezegd worden dat de producten authentiek zijn?}\\*
	\item \textbf{Heeft het aantal gegevens op het netwerk een invloed op het ophalen en wegsturen van informatie?}\\*
	\item \textbf{Welke problemen kunnen (nog) niet opgelost worden met blockchain?}\\*
\end{enumerate}

In de conclusie worden de onderzoeksvragen nog eens overlopen en wordt er hier een antwoord op gegeven.\\*\\*
Dankzij de blockchain technologie kunnen we met zekerheid zeggen dat de authentiteit van producten behouden wordt. Eens de informatie op de blockchain staat, kan die transactie niet meer aangepast worden. Op het netwerk heeft alleen de eigenaar rechten tot zijn producten en kiest dus zelf wat er met gebeurt. Hierdoor kunnen andere bedrijven geen producten gaan aanpassen van andere bedrijven. Dankzij de historiek van de blockchain kan er gezien worden wat er exact met elk product gebeurt, en zo kunnen de eindgebruikers dus ook met zekerheid zien waar hun prouct precies vandaag komt.\\*\\*
Het aantal gegevens heeft dus duidelijk wel een invloed op de snelheid van het ophalen en wegschrijven van informatie. Uit de resultaten kan er afgeleid worden dat het effeciënter wordt als er een grotere lijst wordt ingevoerd. Spijtig genoeg zijn sommige testen niet kunnen doorgaan door het te kort komen van de virtuele machine waarop het netwerk draaide.\\*\\*
Ten slotte werd er gekeken naar wat een blockchain (nog) niet kan oplossen als blockchain gecombineerd wordt met supply chain. Blockchain kan op zich niet direct iets veranderen aan de kwaliteit van de producten, deze verantwoordelijkheid ligt nog altijd bij het bedrijf zelf. Ook is het moeilijk voor de lokale boeren dat niet de middelen hebben om te starten met de integratie met blockchain. Als er gekeken wordt naar China, zijn daar de omstandigheden ook heel moeilijk. 
