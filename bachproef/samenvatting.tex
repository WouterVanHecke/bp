%%=============================================================================
%% Samenvatting
%%=============================================================================

% TODO: De "abstract" of samenvatting is een kernachtige (~ 1 blz. voor een
% thesis) synthese van het document.
%
% Deze aspecten moeten zeker aan bod komen:
% - Context: waarom is dit werk belangrijk?
% - Nood: waarom moest dit onderzocht worden?
% - Taak: wat heb je precies gedaan?
% - Object: wat staat in dit document geschreven?
% - Resultaat: wat was het resultaat?
% - Conclusie: wat is/zijn de belangrijkste conclusie(s)?
% - Perspectief: blijven er nog vragen open die in de toekomst nog kunnen
%    onderzocht worden? Wat is een mogelijk vervolg voor jouw onderzoek?
%
% LET OP! Een samenvatting is GEEN voorwoord!



%%---------- Nederlandse samenvatting -----------------------------------------
%
% TODO: Als je je bachelorproef in het Engels schrijft, moet je eerst een
% Nederlandse samenvatting invoegen. Haal daarvoor onderstaande code uit
% commentaar.
% Wie zijn bachelorproef in het Nederlands schrijft, kan dit negeren, de inhoud
% wordt niet in het document ingevoegd.

\IfLanguageName{english}{%
\selectlanguage{dutch}
\chapter*{Samenvatting}
\lipsum[1-4]
\selectlanguage{english}
}{}

%%---------- Samenvatting -----------------------------------------------------
% De samenvatting in de hoofdtaal van het document

\chapter*{\IfLanguageName{dutch}{Samenvatting}{Abstract}}

Technologie blijft maar evolueren. \\*\\*Bedrijven moeten mee met de tijd, anders worden ze achtergelaten en voorbij gestoken door andere bedrijven. Maar bedrijven hebben niet altijd de tijd / het vermogen om deze nieuwe technologieën te onderzoeken. Dit soort transformatie neemt veel tijd en geld in beslag. In de wereld van supply chain is dit niet anders, en met de opkomst van Blockchain zijn er heel veel mogelijkheden. Daarom is het belangrijk voor bedrijven dat gebruik van maken een supply chain, om deze mogelijkheden goed te onderzoeken om met een zo goed mogelijke uitkomst te komen. In deze bachelorproef is er gefocust op het bouwen van supply chain netwerk op een blockchain en hier bepaalde testen op uit te voeren.\\*\\* Natuurlijk is er eerst een literatuurstudie gedaan om kennis op te doen over aspecten zoals blockchain en supply chain. Alle stappen van literatuurstudie tot het bespreken van resultaten zijn te vinden in deze bachelorproef, zie het einde van de inleiding voor een duidelijke indeling hiervan. De conclusie van het onderzoek is dat het goed mogelijk is om blockchain technologie te combineren met supply chain, ookal is blockchain zelf nog in ontwikkeling en zal nog niet alles lukken. Deze technologie is nog lang niet perfect, dus het is belangrijk om hier aan te blijven werken en blijven onderzoeken. Een uitbreiding van het supply chain netwerk zou een goed begin zijn, het realiseren van een inlog systeem met het dynamisch aanmaken van producten bijvoorbeeld.